\documentclass[final,t]{beamer}

% retrieved from
% http://www-i6.informatik.rwth-aachen.de/~dreuw/latexbeamerposter.php

% {{{ preamble

% Siebel poster printer paper width -- ANSI E: 34"x44"
% Fedex Office: 24"x36" or 36"x48"

\pgfmathsetmacro{\posterheight}{34*2.54}
\pgfmathsetmacro{\posterwidth}{22*2.54}

\usepackage[orientation=landscape,size=custom,width=\posterwidth,height=\posterheight,scale=1.18]{beamerposter}

\mode<presentation>
\usetheme{uiucposter}

% {{{ additional packages

\usepackage{lipsum}
\usepackage[utf8]{inputenc}
%\usepackage{times}
\usepackage{amsmath,amsthm, amssymb, latexsym}
\usepackage[T1]{fontenc}
\usepackage{lmodern}
\usepackage{exscale}
%\usepackage[lighttt]{lmodern}
%\boldmath
\usepackage{booktabs, array}
%\usepackage{rotating} %sideways environment
\usepackage[english]{babel}
\usepackage{wrapfig}
\usepackage[skins,listings]{tcolorbox}

\pgfdeclarelayer{foreground}
\pgfsetlayers{main,foreground}

% }}}

% {{{ tikz

\usepackage{tikz}
\usetikzlibrary{calc}
\usetikzlibrary{positioning}
\usetikzlibrary{fadings}
\usetikzlibrary{chains}
\usetikzlibrary{scopes}
\usetikzlibrary{shadows}
\usetikzlibrary{arrows}
\usetikzlibrary{snakes}
\usetikzlibrary{shapes.misc}
\usetikzlibrary{shapes.symbols}
\usetikzlibrary{shapes.multipart}
\usetikzlibrary{fit}
\usetikzlibrary{shapes.arrows}
\usetikzlibrary{shapes.geometric}
\usetikzlibrary{shapes.callouts}
\usetikzlibrary{decorations.text}

% }}}

% {{{ box config

\tcbset{toplevelbox/.style={%
    enhanced,
    fuzzy shadow={5mm}{-5mm}{0mm}{1mm}{black},
    colback=white,
    fonttitle=\bfseries,
    coltitle=white,
    colframe=uiucblue,
    boxsep=20pt,
    arc=10pt,
    top=0pt,
    boxrule=0pt,
    toptitle=-10pt,
    bottomtitle=-10pt,
    enlarge bottom by=1.25cm,
    titlerule=0pt,
  }
}

% }}}

% {{{ listing config

\tcbset{listing engine=listings}
\tcbset{listingbox/.style={%
    enhanced,
    %boxsep=-9pt,
    %left=15pt,
    arc=7pt,
    enlarge top by=4mm,
    %enlarge bottom by=1mm,
    boxrule=2pt,
  }
}

\definecolor{green}{RGB}{0, 180, 0}

\lstdefinestyle{custompython}{%
  %belowcaptionskip=1\baselineskip,
  breaklines=true,
  frame=none,
  xleftmargin=\parindent,
  language=Python,
  showstringspaces=false,
  basicstyle=\ttfamily,
  keywordstyle=\bfseries\color{uiucblue},
  commentstyle=\itshape\color{purple},
  identifierstyle=\color{black},
  stringstyle=\color{uiuclightblue},
  keywords=[2]{as,True,False},
  keywordstyle=[2]\bfseries\color{green!40!black},
  otherkeywords={>>>,...},
  keywordstyle=[3]\bfseries\color{blue},
  numbers=none,
  columns=flexible,
  rangebeginprefix=\>\>\>\ \#\ ,
  rangeendprefix=\>\>\>\ \#\ ,
  includerangemarker=false,
}

\newtcbinputlisting{\mylisting}[2][]{%
  listing file={#2},
  listingbox,
  listing only,
  listing options={style=custompython,linerange=#1},
}

% }}}

% Display a grid to help align images
%\beamertemplategridbackground[1cm]

% }}}


% {{{ front matter

\title{Poster Template}
\author{Poster Author \texttt{<author@illinois.edu>}}
\institute{%
  Computer Science
  $\cdot$ University of Illinois
}
\date{December 12, 2018}

% }}}

\begin{document}
\begin{frame}[fragile]{}
  \begin{columns}[t]

    % {{{ left column

    \begin{column}{.45\linewidth}
      \begin{tcolorbox}[toplevelbox,adjusted title={Problem Statement}]
        \ Fast multipole method(FMM) is a well known algorithm for N-body simulation. Naïve algorithm would have a complexity of O(N\textsuperscript{2}) because the influence of all other particles have to be calculated for each target particle. FMM reduces the complexity to O(N).However, the efficiency also highly depends on the implementation. 
        \\  Professor Andreas Kloeckner wrote a module called Boxtree for the FMM, which has high-performance parallel GPU implementation to sort all the particles into boxes, which is the first stage of the algorithm. There are also template functions for the rest of the stages, but only a  baseline serial execution path, which would not exploit multicore architecture. A better implementation for is needed to replace the baseline solution.
        
      \end{tcolorbox}

      \begin{tcolorbox}[toplevelbox,adjusted title=Approach]
        \ A task graph integrating Boxtree with a runtime system is proposed as the 
          solution. The task graph is a list of funtions, which are called task. The runtime system can schdule the tasks on different processors so that different tasks can be run simultaneously with dependency specified. The runtime system
          used is called charm4py. charm4py is a high-level parallel and distributed programming framework. It's used for communication between tasks on different
          processors and specifying the execution order. With this integration, it will be able to exploit more of the 
      \end{tcolorbox}

      \begin{tcolorbox}[toplevelbox,adjusted title=Tech Detail 1]
        \ The algorithm group particles into boxes. If a box has more than the specified
        number of patricles, it is further divided into four child boxes. The algorithm first calcualte the multiple expansion for all child-less boxes. It use these
        multipole expansion to calcualte the multipole expansion for parent level boxes.
        Interaction between adjacent boxes are directly calcualted, interaction between
        non-adjacent boxes can be calculated by using those multipole expansions to calculate local expansions. The algorithm is very complicated, this is only a brief introduction.
        \\ Different approaches were conducted in order to exploit the multicore machine.
        First step was to identify the non-dependent steps in the algorithm. It was clear that calculating the direct interaction is not dependent on any other stages other
        than the sorting stage. Therefore it becomes a independent task which is started at very beginning. All the other stages depends on The second stage, which is to calculate the multipole expansion of all boxes. So step2 is the second task to start. Stage 5 is evaluating multipole expansion for some boxes, which is also independent. Step 4,6 are for calculating the local expansions, which can be separated. Stage 7 is using the result of step 4,6 to calculate the local expansion for smaller boxes. Charm4py has a feature called reduction which can
        collect output through different processes and process only after receives all
        the dependencies. So 7 can be executed after 4 and 6 finishes. Finally stage
        8 runs after all the other stages to evaluate local expansions and sums everything
        up.
      \end{tcolorbox}

    \end{column}

    % }}}

    % {{{ right column

    \begin{column}{.45\linewidth}
      \begin{tcolorbox}[toplevelbox,adjusted title=Tech Detail 2]
        \lipsum[5]
      \end{tcolorbox}

      \begin{tcolorbox}[toplevelbox,adjusted title=Results]
        \lipsum[6-7]
      \end{tcolorbox}

      \begin{tcolorbox}[toplevelbox,adjusted title=References]
        \lipsum[8]
      \end{tcolorbox}

    \end{column}

    % }}}

  \end{columns}
\end{frame}

\end{document}

% vim: foldmethod=marker
